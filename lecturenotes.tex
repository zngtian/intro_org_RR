% Created 2017-01-29 Sun 21:43
% Intended LaTeX compiler: pdflatex
\documentclass[11pt]{article}
\usepackage[utf8]{inputenc}
\usepackage[T1]{fontenc}
\usepackage{graphicx}
\usepackage{grffile}
\usepackage{longtable}
\usepackage{wrapfig}
\usepackage{rotating}
\usepackage[normalem]{ulem}
\usepackage{amsmath}
\usepackage{textcomp}
\usepackage{amssymb}
\usepackage{capt-of}
\usepackage{hyperref}
\usepackage[margin=1in]{geometry}
\date{}
\title{Workshop on Emacs, Org-Mode, and Reproducible Research}
\hypersetup{
 pdfauthor={},
 pdftitle={Workshop on Emacs, Org-Mode, and Reproducible Research},
 pdfkeywords={},
 pdfsubject={},
 pdfcreator={Emacs 25.1.1 (Org mode 9.0.3)},
 pdflang={English}}
\begin{document}

\maketitle
\section*{Workshop description}
\label{sec:org0e62130}

Reproducible research is the idea that data analyses, and more
generally, scientific claims, are published with their data and
software code so that others may verify the findings and build upon
them.\footnote{This definition is from
\url{https://www.coursera.org/learn/reproducible-research}.} Reproducible research consists of a workflow that
incorporates narrative description, mathematical proof, code
development, result display in either tables or graphs, and
bibliography management. Handy tools can make implementing
reproducible research easier. Among many good alternatives, such as R
Markdown and knitr that are included in RStudio and IPython Notebook,
this workshop introduces Emacs and its org-mode.

Emacs is a text editor whose functions are extensible through its
contributed packages, among which the org-mode is one of the most
widely used. Within the org-mode, we can write structured text, write
mathematical equations as in \LaTeX{}, and execute codes in R, Python and
other programming languages, from which the results are dynamically
embedded in the file in tables and graphs. The org-mode file can be
easily exported to either a PDF or HTML file, in which section
numbers, cross-references, mathematics, tables, and graphs are
properly inserted. With the help of other modes in Emacs, we can use
the org-mode as bibliography management, task management, daily
planner, and a list goes on.

This workshop is to share my experience of using org-mode to take
notes, write papers, and manage my daily work through the perspective
of reproducible research. I hope that audiences will start to use
Emacs and org-mode after attending this workshop, and explore their
plentiful and powerful functions by continuously playing with them
every day.

\section*{Instructor: Zheng Tian}
\label{sec:orgda8e3bb}

Zheng Tian is Assistant Professor at International School of Economics and
Business, Capital University of Economics and Business, Beijing,
China. He is currently visiting the Regional Research Institute where
he was a former research assistant from 2009 to 2014.


\section*{Time and Location}
\label{sec:orgc62bfe4}

February 9th, Time TBA, at Regional Research Institute.
\end{document}